\documentclass[]{article}

\usepackage{listings}
\usepackage{color}
\usepackage{placeins}


\definecolor{dkgreen}{rgb}{0,0.6,0}
\definecolor{gray}{rgb}{0.5,0.5,0.5}
\definecolor{mauve}{rgb}{0.58,0,0.82}

\lstset{frame=tb,
  language=C++,
  aboveskip=3mm,
  belowskip=3mm,
  showstringspaces=false,
  columns=flexible,
  basicstyle={\small\ttfamily},
  numbers=left,
  numberstyle=\color{gray},
  keywordstyle=\color{blue},
  commentstyle=\color{dkgreen},
  stringstyle=\color{mauve},
  breaklines=true,
  breakatwhitespace=true,
  tabsize=4
}

\setlength{\parindent}{0pt}
\setlength{\tabcolsep}{.5 cm}

%opening
\title{Causes for Eviction Notices in Assorted Neighborhoods of San Francisco}
\author{
Aitken, Connor\\
\texttt{connor.aitken500@gmail.com}
\and
Cheung, Leon\\
\texttt{leonjcheung@gmail.com}
}

\begin{document}

\maketitle

\begin{abstract}
In recent years, the gentrification of San Francisco has become an increasingly controversial topic for those living in the Bay Area. We have set out to investigate whether the evictions we hear about in the news are morally justifiable or a result of landlords looking to generate greater revenues on their property, due to the great demand for housing. We pulled data from the city's open database initiative \textit{dataSF} and used the C++ Programming Language to statistically analyze the data. We found that there is a veritably greater rate in evictions that we perceive to be motivated by greater potential profits, especially in neighborhoods which are exploding due to the rise of tech companies wishing to maintain offices in San Francisco. With this result in mind, we gain a greater perspective on the objective state of the state of housing in San Francisco.
\end{abstract}

\section{Data Collection}
We downloaded data on eviction statistics from SF OpenData and counted each of the reasons for eviction with a program we designed and created using C++. For some of these expected counts, the data was less than five so we were unable to incorporate it in our testing. After separating the data into eleven districts based off location, we performed $\chi ^{2}$ Data Tests for Homogeneity using the valid counts.
\section{Data Context}
\newline \newline
Categories: \newline \newline
\begin{tabular}{l | l}

	Tenant Action    	& nonpayment, breach, nuisance, illegal,\\& unapproved subtenant, latepay \\							 	\\ 
	Landlord Action     & failsignrenew, owner move in, capital improvement, \\ & substantial rehab, roommate same unit 		\\				 		\\
	Development    		& demolition, ellis act withdrawal, condo \\ & conversion 			\\					\\ 
	Just Cause    		& nonpayment, latepayment, breach of lease, \\ &  ownermovein, capitalimprovement, ellisactwithdrawal, \\ & nuisance, illegal, demolition 				 			\\	
	\end{tabular}

\section{Analysis}
\subsection{Summary Statistics}
This table and section is for testing.

\section{Inferential Procedures}
\subsection{District 1}

\begin{table}[h]
\centering
\begin{tabular}{l | l}
Neighborhoods & Inner Richmond, Lone Mountain, Outer Richmond \\
\end{tabular}
\end{table}
\FloatBarrier

\begin {table}[h]
\centering
\begin{tabular}{l | l | l | l}
	
	Reason	&  $\chi ^{2}$ & df & p-Value \\
	Tenant Action 		   &  15.1570  & 10  & 0.1264  \\
	Landlord Action	       &  10.4669  & 6   & 0.1062 \\
	Development			   &  0.9854   & 2   & 0.6110 \\
	Just Cause Removal	   &  39.9610  & 16  & 0.0008 \\
\end{tabular} \newline
\end{table}
\FloatBarrier

For District 1, there was homogeneity shown in tenant action, landlord action, and development, however there was not homogeneity shown in just cause removal. 

how to illustrate p values, df, chi statistic in beautiful way

\subsection{District 2}


\begin{table}[h]
	\centering
	\begin{tabular}{l | l}
		Neighborhoods & Seacliff, Presidio Heights, Marina, Russian Hill, Pacific Heights \\
	\end{tabular}
\end{table}
\FloatBarrier

\begin {table}[h]
\centering
\begin{tabular}{l | l | l | l}
	
	Reason	&  $\chi ^{2}$ & df & p-Value \\
	Tenant Action 		 & 45.4120 &  15   & 0.0001 		\\
	Landlord Action	     & 11.8037 &  6     &  	0.0665	\\
	Development			 & 0.3398 &  6    &  0.9993		\\
	Just Cause Removal	 & 127.1146   &  24 &  	0.0000		\\
\end{tabular} \newline
\end{table}
\FloatBarrier

\subsection{District 3}

\begin{table}[h]
	\centering
	\begin{tabular}{l | l}
		Neighborhoods & North Beach, Nob Hill \\
	\end{tabular}
\end{table}
\FloatBarrier

\begin {table}[h]
\centering
\begin{tabular}{l | l | l | l}
	
	Reason	&  $\chi ^{2}$ & df & p-Value \\
	Tenant Action 		  & 8.0298 &  5  &  0.1955    \\
	Landlord Action	      & 2.2699 &  3  &  0.7037    \\
	Development			  & 0.0000 &  1  &  1.0000    \\
	Just Cause Removal	  & 19.6549 &  8  &  0.0117    \\
\end{tabular} \newline
\end{table}
\FloatBarrier

\subsection{District 5}

\begin{table}[h]
	\centering
	\begin{tabular}{l | l}
		Neighborhoods & Haight Ashbury, Hayes Valley, Western Addition \\
	\end{tabular}
\end{table}
\FloatBarrier

\begin {table}[h]
\centering
\begin{tabular}{l | l | l | l}
	
	Reason				&  $\chi ^{2}$ & df & p-Value \\
	Tenant Action 		   &  32.491   &  10 & 0.0003 \\
	Landlord Action	       &  10.671  & 6  & 0.0991 \\
	Development			   &  0.9271  &  4 & 0.9206 \\
	Just Cause Removal	   &  111.3054  & 16  & 0.0000 \\
\end{tabular} \newline
\end{table}
\FloatBarrier

\subsection{District 6}


\begin{table}[h]
	\centering
	\begin{tabular}{l | l}
		Neighborhoods & Tenderloin, South of Market, Financial District/South Beach, Mission Bay  \\
	\end{tabular}
\end{table}
\FloatBarrier

\begin {table}[h]
\centering
\begin{tabular}{l | l | l | l}
	
	Reason				 &  $\chi ^{2}$ & df & p-Value \\
	Tenant Action 		   &  46.9304  & 4  & 0.0000 \\
	Landlord Action	       &  0.0000  & -2  & 1.0000 \\
	Development			   &  0.0000  & -2  & 1.0000 \\
	Just Cause Removal	   &  63.8182  & 4  & 0.0000 \\
\end{tabular} \newline
\end{table}
\FloatBarrier

\subsection{District 8}

\begin{table}[h]
	\centering
	\begin{tabular}{l | l}
		Neighborhoods & Noe Valley, Glen Park, Twin Peaks
	\end{tabular}
\end{table}
\FloatBarrier

\begin {table}[h]
\centering
\begin{tabular}{l | l | l | l}
	
	Reason				 &  $\chi ^{2}$ & df & p-Value \\
	Tenant Action 		   &  10.5992  &  2 & 0.0050 \\
	Landlord Action	       &  1.1193  &  0 & 1.0000 \\
	Development			   &  1.1314  & 0  & 1.0000 \\
	Just Cause Removal	   &  37.6297  &  6 & 0.0000 \\
\end{tabular} \newline
\end{table}
\FloatBarrier

\subsection{District 9}

\begin {table}[h]
\centering
\begin{tabular}{l | l}
	Neighborhoods &  Portola, Bernal Heights/ Bernal North/ Bernal South, Mission/ Inner Mission
\end{tabular}
\begin{tabular}{l | l | l | l}
	
	Reason				 &  $\chi ^{2}$ & df & p-Values\\
	Tenant Action 		   & 12.4686   &  6 & 0.0523 \\
	Landlord Action	       &  1.3052  & 2  & 0.5207 \\
	Development			   &  19.1170  & 0  & 1.0000 \\
	Just Cause Removal	   &  133.8546  & 12  & 0.0000 \\
\end{tabular} \newline
\end{table}
\FloatBarrier

\subsection{District 10}

\begin {table}[h]
\centering
\begin{tabular}{l | l}
	Neighborhoods & Visitacion valley/Bayview Heights, Bayview Valley, Huner's Point, Portero Hill
\end{tabular}
\begin{tabular}{l | l | l | l}
	
	Reason				 &  $\chi ^{2}$ & df & p-Value \\
	Tenant Action 		   &  3.8906  &  4  & 0.4210 \\
	Landlord Action	       &  1.5766  &  1  & 0.3173 \\
	Development			   &  0.0000  &  -1  & 1.0000 \\
	Just Cause Removal	   &  8.6997  &  5  & 0.1640 \\
\end{tabular} \newline
\end{table}
\FloatBarrier



\subsection{District 11}

\begin{table}[h]
	\centering
	\begin{tabular}{l | l}
		Neighborhoods &  Oceanview/Merced/Ingleside, Outer Mission, Excelsior  \\
	\end{tabular}
\end{table}
\FloatBarrier

\begin {table}[h]
\centering
\begin{tabular}{l | l | l | l}
	
	Reason				 &  $\chi ^{2}$ & df & p-Value \\
	Tenant Action 		   &  23.1291  &  8  & 0.0032 \\
	Landlord Action	       &   2.2277 &  2  & 0.3283 \\
	Development			   &  3.8359  &  2  & 0.1469 \\
	Just Cause Removal	   &  30.6304  &  14  & 0.0062 \\
\end{tabular} \newline
\end{table}
\FloatBarrier

\subsection{District 3 vs. District 6}
\begin{table}[h]
\centering
\begin{tabular}{l | l}
Neighborhoods &  North Beach, Nob Hill, 			 \\ 
			  &  Tenderloin, South of Market, Financial Ditsrict/South Beach \\
\end{tabular}
\end{table}

\begin {table}[h]
\centering
\begin{tabular}{l | r | r | r}	
Reason				 &  $\chi ^{2}$ & df    & p-Value   \\
Tenant Action 		 &  50.260      &  8    & 0.0000    \\
Landlord Action	     &  na          &  na   & na        \\
Development			 &  na          &  na   & na        \\
Just Cause Removal	 &  341.925     &  12   & 0.0000    \\
\end{tabular} \newline
\end{table}

The Landlord Action and Development categories yielded no results because we were unable to perform the test, due to the expected counts condition. We were successful in running a test that produced clear results for the Tenant Action and Just Cause Removal categories, however. The greatest component of the Tenant Action category was the Non Payment reason for District 6, which we think is likely due to rising rents in the Tenderloin neighborhood.
\newline\newline
We saw clear non-homogeneity in the Just Cause Removal test, where the p-Value approached zero. Here, the largest components for both District 3 and District 6 were due to the Ellis Act Withdrawal eviction reason, with components of 144.9395 and 81.4898, respectively. This indicates that an immense amount of landlords are removing their property from the rental market for other uses.

\subsection{District 1 vs. District 4}
\begin{table}[!h]
\centering
\begin{tabular}{l | l}
Neighborhoods &  Outer Richmond, Inner Richmond, Lone Mountain/USF,  \\ 
			  &  Sunset/Parkside \\
\end{tabular}
\end{table}

\begin {table}[!h]
\centering
\begin{tabular}{l | r | r | r}	
Reason				 &  $\chi ^{2}$ & df    & p-Value   \\
Tenant Action 		 &  13.3248     & 12    & 0.3459    \\
Landlord Action	     &  8.6267      & 6     & 0.1957    \\
Development			 &  78.7158     & 3     & 0.0000    \\
Just Cause Removal	 &  106.561     & 24    & 0.0000    \\
\end{tabular} \newline
\end{table}

Between these two districts, we see that the evictions due to renter and landlord action occur at relatively the same rate relative to the size of each district. It may be interesting to note that for the Landlord Action category, much of the statistic was made of one component: the Capital Improvement reason for District 4 at 5.2804, which suggests that in the Sunset/Parkside neighborhood landlords are more likely to make significant improvements on their apartments, which temporarily evict tenants from their rooms.
\newline\newline
For reasons which we grouped under Development, these produced components with large values. The greatest contributor to the Development reason was from the Demolition reason from District 4 at 39.8452. This altogether may not be too unexpected, because, from anecdotal experience, there are many buildings which have fallen into disarray or may not be within building code in the first place.
\newline\newline
When we study the Just Cause Removal test, we find more clear differences between these two districts. Both of the largest components of the test statistic resulted from District 4, in the Demolition and Ellis Act Withdrawal reasons, at 47.078 and 15.015 respectively. 

\subsection{District 3 vs. District 5}
\begin{table}[!h]
\centering
\begin{tabular}{l | l}
Neighborhoods &  North Beach, Nob Hill,  \\ 
			  &  Haight Ashbury, Hayes Valley, Western Addition \\
\end{tabular}
\end{table}

\begin {table}[!h]
\centering
\begin{tabular}{l | r | r | r}	
Reason				 &  $\chi ^{2}$ & df    & p-Value   \\
Tenant Action 		 &  9.7614      & 16    & 0.8788    \\
Landlord Action	     &  0.9936      & 8     & 0.9983    \\
Development			 &  0.5791      & 4     & 0.9654    \\
Just Cause Removal	 &  117.5088    & 28    & 0.0000    \\
\end{tabular} \newline
\end{table}

Upon examining the p-Values for each of these tests, it is immediately apparent that these two districts are quite homogenous to each other, at p-Values above 0.85 for Tenant Action, Landlord Action, and Development. However, we see a departure from homogeneity in the Just Cause Removal test, which we could see as a unification of all three of these tests.
\newline\newline
As we parse the components of the Just Cause Removal test, we find that three key components make up the bulk of the test statistic. For District 3, it is the Ellis Act Withdrawal reason, at a value of 39.0179, and for District 5, it is the Owner Move In and Capital Improvement reasons at 30.4084 and 21.7573 respectively that contribute the most. Perhaps District 5 is an attractive place for landlords to find to live in their apartment property, and it is possible that in District 3, landlords are liquidating their property in anticipation of higher profits outside of the rental market.

\subsection{District 5 vs. District 6}
\begin{table}[!h]
\centering
\begin{tabular}{l | l}
Neighborhoods & Haight Ashbury, Hayes Valley, Western Addition\\
			  & Tenderloin, South of Market  \\ 
\end{tabular}
\end{table}

\begin {table}[!h]
\centering
\begin{tabular}{l | r | r | r}	
Reason				 &  $\chi ^{2}$ & df    & p-Value   \\
Tenant Action 		 &  45.3672     & 10    & 0.0000    \\
Landlord Action	     &  na          & na    & na        \\
Development			 &  na          & na    & na        \\
Just Cause Removal	 &  407.8489    & 20    & 0.0000    \\
\end{tabular} \newline
\end{table}

For Landlord Action and Development, we could not run the tests because we had expected counts less than 5.
\newline\newline
In the Tenant Action test, we find a clear departure from homogeneity between District 5 and District 6, suggesting a difference in the type of renters between these two areas. Indeed, we can see that District 6 contributes much to the final test statistic, with the two reasons Non Payment and Nuisance at 16.0512 and 21.4209 respectively, suggesting that it may not be advantageous for landlords to hold a property in this area relative to District 5.
\newline\newline
In the Just Cause Removal test, the largest departures from homogeneity are revealed in the components Owner Move In of District 5, and Nuisance and Owner Move In of District 6, at 133.2342, 68.6161, and 61.3065.

\section{Conclusion}
sdfsdafasdf
\newpage
\appendix
\section{Source code}
\begin{lstlisting}
#include <iostream>

int main()
{
	std::cout < S"dF" << SDF"";;endl;
}
\end{lstlisting}

\section{Data source}

\end{document}
