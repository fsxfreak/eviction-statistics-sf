\documentclass[]{article}

\usepackage{listings}
\usepackage{color}
\usepackage{placeins}


\definecolor{dkgreen}{rgb}{0,0.6,0}
\definecolor{gray}{rgb}{0.5,0.5,0.5}
\definecolor{mauve}{rgb}{0.58,0,0.82}

\lstset{frame=tb,
  language=C++,
  aboveskip=3mm,
  belowskip=3mm,
  showstringspaces=false,
  columns=flexible,
  basicstyle={\small\ttfamily},
  numbers=left,
  numberstyle=\color{gray},
  keywordstyle=\color{blue},
  commentstyle=\color{dkgreen},
  stringstyle=\color{mauve},
  breaklines=true,
  breakatwhitespace=true,
  tabsize=4
}

\setlength{\parindent}{0pt}
\setlength{\tabcolsep}{.5 cm}

%opening
\title{Causes for Eviction Notices in Assorted Neighborhoods of San Francisco}
\author{
Aitken, Connor\\
\texttt{connor.aitken500@gmail.com}
\and
Cheung, Leon\\
\texttt{leonjcheung@gmail.com}
}

\begin{document}

\maketitle

\begin{abstract}
Renter eviction leads to the displacement of families and an influx of people at the homeless shelters.  
\end{abstract}

\section{Data Collection}
We downloaded data on eviction statistics from SF OpenData and counted each of the reasons for eviction with a program we designed and created using C++. For some of these expected counts, the data was less than five so we were unable to incorporate it in our testing. After separating the data into eleven districts based off location, we performed $\chi ^{2}$ Data Tests for Homogeneity using the valid counts.
\section{Data Context}
here we explain what the various eviction reasons are, and show maps of San Francisco containing all the neighborhoods split up into regions
\newline \newline
Categories: \newline \newline
\begin{tabular}{l | l}

	Tenant Action    	& nonpayment, breach, nuisance, illegal,\\& unapproved subtenant, latepay \\							 	\\ 
	Landlord Action     & failsignrenew, owner move in, capital improvement, \\ & substantial rehab, roommate same unit 		\\				 		\\
	Development    		& demolition, ellis act withdrawal, condo \\ & conversion 			\\					\\ 
	Just Cause    		& nonpayment, latepayment, breach of lease, \\ &  ownermovein, capitalimprovement, ellisactwithdrawal, \\ & nuisance, illegal, demolition 				 			\\	
	\end{tabular}

can also talk about the legal issues with the landlord tenant relationship
\section{Analysis}
\subsection{Summary Statistics}
This table and section is for testing.

\section{Inferential Procedures}
\subsection{District 1}

\begin{table}[!h]
\centering
\begin{tabular}{l | l}
Neighborhoods & Inner Richmond, Lone Mountain, Outer Richmond \\
\end{tabular}
\end{table}
\FloatBarrier

\begin {table}[!h]
\centering
\begin{tabular}{l | l | l | l}
	
	Reason	&  $\chi ^{2}$ & df & p-Value \\
	Tenant Action 		   &  15.1570  & 10  & 0.1264  \\
	Landlord Action	       &  10.4669  & 6   & 0.1062 \\
	Development			   &  0.9854   & 2   & 0.6110 \\
	Just Cause Removal	   &  39.9610  & 16  & 0.0008 \\
\end{tabular} \newline
\end{table}
\FloatBarrier


For District 1, there was homogeneity shown in all areas besides "Just Cause Removal". This may be because of Lone Mountain's low nonpay component of 0.4827 or Inner Richmond's high ownermovein component of 4.1747.


\subsection{District 2}
\begin{table}[!h]
	\centering
	\begin{tabular}{l | l}
		Neighborhoods & Seacliff, Presidio Heights, Marina, Russian Hill, Pacific Heights \\
	\end{tabular}
\end{table}
\FloatBarrier

\begin {table}[!h]
\centering
\begin{tabular}{l | l | l | l}
	
	Reason	&  $\chi ^{2}$ & df & p-Value \\
	Tenant Action 		 & 45.4120 &  15   & 0.0001 		\\
	Landlord Action	     & 11.8037 &  6     &  	0.0665	\\
	Development			 & 0.3398 &  6    &  0.9993		\\
	Just Cause Removal	 & 127.1146   &  24 &  	0.0000		\\
\end{tabular} \newline
\end{table}
\FloatBarrier

For District 2, homogeneity was only shown in "Just Cause Removal" and "Tenant Action." Pacific Heights' high nonpayment component of 11.6593 likely impacts both of these sections.

\subsection{District 3}

\begin{table}[!h]
	\centering
	\begin{tabular}{l | l}
		Neighborhoods & North Beach, Nob Hill \\
	\end{tabular}
\end{table}
\FloatBarrier

\begin {table}[!h]
\centering
\begin{tabular}{l | l | l | l}
	
	Reason	&  $\chi ^{2}$ & df & p-Value \\
	Tenant Action 		  & 8.0298 &  5  &  0.1955    \\
	Landlord Action	      & 2.2699 &  3  &  0.7037    \\
	Development			  & 0.0000 &  1  &  1.0000    \\
	Just Cause Removal	  & 19.6549 &  8  &  0.0117    \\
\end{tabular} \newline
\end{table}
\FloatBarrier

For District 4, Homogeneity was shown in all categories but "Just Cause Removal." This may be because of North Beach's almost double nonpayment and ellisactwithdrawl compared to Nob Hill's components.

\subsection{District 5}

\begin{table}[!h]
	\centering
	\begin{tabular}{l | l}
		Neighborhoods & Haight Ashbury, Hayes Valley, Western Addition \\
	\end{tabular}
\end{table}
\FloatBarrier

\begin {table}[!h]
\centering
\begin{tabular}{l | l | l | l}
	
	Reason				&  $\chi ^{2}$ & df & p-Value \\
	Tenant Action 		   &  32.491   &  10 & 0.0003 \\
	Landlord Action	       &  10.671  & 6  & 0.0991 \\
	Development			   &  0.9271  &  4 & 0.9206 \\
	Just Cause Removal	   &  111.3054  & 16  & 0.0000 \\
\end{tabular} \newline
\end{table}
\FloatBarrier

Homogeneity is shown in all categories besides "Tenant Action" and "Just Cause Removal." In regards to "Tenant Action," Hayes Valley high nonpayment component of 6.8882 and Western Addition's high nuisance component of 6.3202 likely contribute to the lack of homogeneity, and in "Just Cause Removal," Western Addition's extremely low capital improvement component of 0.1674 does not correlate with the next lowest in Western Addison of 14.6997.

\subsection{District 6}


\begin{table}[!h]
	\centering
	\begin{tabular}{l | l}
		Neighborhoods & Tenderloin, South of Market, Financial District/South Beach, Mission Bay  \\
	\end{tabular}
\end{table}
\FloatBarrier

\begin {table}[!h]
\centering
\begin{tabular}{l | l | l | l}
	
	Reason				 &  $\chi ^{2}$ & df & p-Value \\
	Tenant Action 		   &  46.9304  & 4  & 0.0000 \\
	Landlord Action	       &  0.0000  & -2  & 1.0000 \\
	Development			   &  0.0000  & -2  & 1.0000 \\
	Just Cause Removal	   &  63.8182  & 4  & 0.0000 \\
\end{tabular} \newline
\end{table}
\FloatBarrier

Homogeneity is shown in the categories of "Tenant Action" and "Just Cause Removal." This may be because of Financial District/South Beach's high nuisance component of 8.1637 and its low nonpayment component of 6.3031.

\subsection{District 8}

\begin{table}[!h]
	\centering
	\begin{tabular}{l | l}
		Neighborhoods & Noe Valley, Glen Park, Twin Peaks
	\end{tabular}
\end{table}
\FloatBarrier

\begin {table}[!h]
\centering
\begin{tabular}{l | l | l | l}
	
	Reason				 &  $\chi ^{2}$ & df & p-Value \\
	Tenant Action 		   &  10.5992  &  2 & 0.0050 \\
	Landlord Action	       &  1.1193  &  0 & 1.0000 \\
	Development			   &  1.1314  & 0  & 1.0000 \\
	Just Cause Removal	   &  37.6297  &  6 & 0.0000 \\
\end{tabular} \newline
\end{table}
\FloatBarrier

Homogeneity is shown in the categories of "Tenant Action" and "Just Cause Removal." The reasons for this in "Tenant Action" may be Glen Park's low nuisance component of 0.2598, and in "Just Cause Removal" it may be because of Glen Park's high breach value of 5.9563 and low ellisactwithdrawl of 0.0003.

\subsection{District 9}

\begin {table}[!h]
\centering
\begin{tabular}{l | l}
	Neighborhoods &  Portola, Bernal Heights/ Bernal North/ Bernal South, Mission/ Inner Mission
\end{tabular}
\begin{tabular}{l | l | l | l}
	
	Reason				 &  $\chi ^{2}$ & df & p-Values\\
	Tenant Action 		   & 12.4686   &  6 & 0.0523 \\
	Landlord Action	       &  1.3052  & 2  & 0.5207 \\
	Development			   &  19.1170  & 0  & 1.0000 \\
	Just Cause Removal	   &  133.8546  & 12  & 0.0000 \\
\end{tabular} \newline
\end{table}
\FloatBarrier

Homogeneity is shown in all areas besides "Just Cause Removal." This is most likely because Portola has an exremently high ellisactwithdrawl value of 20.7812. The next highest is only 4.3994 in Mission.  

\subsection{District 10}

\begin {table}[!h]
\centering
\begin{tabular}{l | l}
	Neighborhoods & Visitacion valley/Bayview Heights, Bayview Valley, Huner's Point, Portero Hill
\end{tabular}
\begin{tabular}{l | l | l | l}
	
	Reason				 &  $\chi ^{2}$ & df & p-Value \\
	Tenant Action 		   &  3.8906  &  4  & 0.4210 \\
	Landlord Action	       &  1.5766  &  1  & 0.3173 \\
	Development			   &  0.0000  &  -1  & 1.0000 \\
	Just Cause Removal	   &  8.6997  &  5  & 0.1640 \\
\end{tabular} \newline
\end{table}
\FloatBarrier

District 10 has evidence of homogeneity. Each of the reason's p-Value is above 0.05.

\subsection{District 11}

\begin{table}[!h]
	\centering
	\begin{tabular}{l | l}
		Neighborhoods &  Oceanview/Merced/Ingleside, Outer Mission, Excelsior  \\
	\end{tabular}
\end{table}
\FloatBarrier

\begin {table}[!h]
\centering
\begin{tabular}{l | l | l | l}
	
	Reason				 &  $\chi ^{2}$ & df & p-Value \\
	Tenant Action 		   &  23.1291  &  8  & 0.0032 \\
	Landlord Action	       &   2.2277 &  2  & 0.3283 \\
	Development			   &  3.8359  &  2  & 0.1469 \\
	Just Cause Removal	   &  30.6304  &  14  & 0.0062 \\
\end{tabular} \newline
\end{table}
\FloatBarrier

Only "Tenant Action" and "Just Cause Removal" do not show evidence of homogeneity in District 11. For "Tenant Action," this may be because of Outer Mission's high breach component of 5.7766, and in "Just Cause Removal" it could be because of the Oceanview/Merced/Ingleside's nuisance component of 0.0220.



\section{Conclusion}
sdfsdafasdf
\newpage
\appendix
\section{Source code}
\begin{lstlisting}
#include <iostream>

int main()
{
	std::cout < S"dF" << SDF"";;endl;
}
\end{lstlisting}

\section{Data source}

\end{document}
